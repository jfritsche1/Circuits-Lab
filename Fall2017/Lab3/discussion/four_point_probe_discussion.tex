The measured resistivity is on the order of $10^{-2}$, quite a small resistivity. The theory used to calculate the resistivity is valid because the distances between probes is negligible in comparison to the dimensions of the wafer and the wafer is extremely thin in comparison to the other measured lengths. Using this resistivity value, the impurity concentration is determined to be quite high, around $2 \times 10^{17} [\frac{donors}{cm^3}]$. The high impurity concentration explains why the resistivity is so low. When a large number of donors are present in silicon, more carrier electrons are generated in the conduction band. Thus, for the same potential difference, more charges can move around, and thus a higher current is possible. Therefore, the resistivity is expected to be low. However, a more thorough analysis is required before the impurity concentration figure is finalized. More data points and a linear regression analysis is required to determine a more accurate value for the resistivity and therefore the carrier concentration.
