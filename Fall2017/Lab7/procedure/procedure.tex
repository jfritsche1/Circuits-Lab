The objective of this lab is to design a transistor inverter and observe both the switching and amplifier characteristics of MOSFETs. For the first circuit, the switching behavior of a MOSFET is observed. A circuit with multiple components is built and a voltage pulse is given to the transistor to observe the output voltage signal. The input pulse signal is increased so both the turn-on and turn-off transients are observed. Both the input and output signals are to be observed and the time delay is measured. Next, the circuit for the MOSFET Amplifier is built. The proper resistor and capacitor are to be chosen to determine the lower -3db cutoff frequency, while the transistor determines the upper -3dB cutoff frequency. The bias voltage is measured without connecting the input source to the amplifier, which should be half of the DC power supply. After measuring, apply an input signal at 10kHz frequency to display both the input and output voltage signal (sinusoidal). The input voltage is increased until the output voltage starts to distort. The peak output voltage gain is measured and is compared to the calculated voltage gain. The function generator is then set to CW mode and the different frequencies around the cutoff are measured along with voltage gain. Lastly, the function generator is set to pulse mode and the frequency is adjusted to measure the cut-off frequency. The input and output signals are measured and observed to see if there are any distortions. \\
