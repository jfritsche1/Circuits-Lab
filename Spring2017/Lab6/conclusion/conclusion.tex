\subsection{CMOS Inverter Circuit}

The CMOS inverter is an inverter architecture that is optimized for noise and power consumption. This is because of the fact that inverter's output fully reaches supply and ground in either case. Propagation delays acquired are on the order of about $100$\si{\pico\second}.

\subsection{CMOS Inverter Layout}

Level 1, the simplest of all of the MOSFET models, is closest to the SPICE model typically used for simulations. Increasing the widths of transistors tends to increase the operating speed of the inverter. When the inverter is loaded, its propagation delay is seen to increase. In the $1.2$\si{\micro\meter} NMOS - $3.0$\si{\micro\meter} PMOS and $0.6$\si{\micro\meter} NMOS - $1.5$\si{\micro\meter} PMOS tests, the voltage transfer characteristic is shifted to the right from the $0.6$\si{\micro\meter} NMOS and PMOS test. This is because the transistors are more balanced in that their transconductance parameters are closer. When the widths are equal, because the mobility of the carrier holes in the PMOS is much lower, the PMOS's transconductance parameter is much lower, causing the transistors to be unbalanced and for the voltage transfer characteristic to not be as centered.
