The purpose of the current mirror is to take an input current $i_{in}$ and produce an identical output current $i_{out}$. Changing the width of transistors in the current mirror simply increases the currents when widened or decreases the currents when narrowed. By carefully controlling the widths of different transistors, it is conceivable that an arbitrary output current whose magnitude is proportional to the input current can be produced. \\

The common source amplifier with the PMOS current mirror operates like a typical common-source amplifier. Its gain turns out to be rather high for a common-source amplifier, which may have to do with the addition of the PMOS current mirror. The simulation gain ends up being higher than the calculated gain. The reasons for this are unclear and likely have to do with the inner workings of the MOSFET model and not one of the specified parameters since different parameter values were defaulted and yielded similar results.
