The common source amplifier exhibited the behavior of a logical inverter with low input voltage yielding high output voltage and high input voltage yielding low input voltage. At $V_{in}$ lower than threshold, the transistor operates in cutoff, and $V_{out}$ is therefore at its highest value. When $V_{in}$ exceeds threshold, current begins to flow which causes a voltage drop accros the resistor, thus decreasing $V_{out}$. The current continues to increase as $V_{in}$ increases, resulting in larger voltage drops across the resistor which further decreases $V_{out}$. Eventually, the transistor hits triode mode and current levels off. The bias point, like the common drain amplifier, occurs in the middle of the saturation region and corresponds to the steepest slope in the curve.    
