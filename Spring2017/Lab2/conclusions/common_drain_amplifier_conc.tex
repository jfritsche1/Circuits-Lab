% NMOS
The resulting measurements from the NMOS transistor experiements agreed well with theory, and we were able to see how the transistor moved throught all the modes of operation. The data clearly showed the transistor's cutoff voltage and threshold voltage. We noticed the effect of the NMOS's capacitance when moving from triode mode to cutoff. The accumulated charge was enough to keep the channel open so that the drain current was twice as large as when the accumulated charge was not present. 
  
% PMOS
The PMOS transistor clearly displayed operation in cutoff and triode mode. We were able to observe a variable cutoff voltage, because the source voltage is not grounded. We noticed that we were able to change the cutoff voltage by varying the gate voltage.
% COMMON DRAIN AMP
The common drain amplifier essentially follows the trend of the input voltage at the gate, $V_{in}$. It starts in the cutoff region since the input voltage is too low to form the channel. However, as the input is increased, the current flows, and an output voltage develops over the resistor. It transitions from cutoff to saturation to triode, but does not reach triode in the range of voltages considered. The proper bias point occurs in the middle of the saturation region since the slope is steepest. Moreover, biasing the amplifier in the saturation region leads to high linearity in the output of the amplifier.

% COMMON SOURCE AMP
