The common drain amplifier essentially follows the trend of the input voltage at the gate, $V_{in}$. It starts in the cutoff region since the input voltage is too low to form the channel. However, as the input is increased, the current flows, and an output voltage develops over the resistor. The voltage over the resistor is directly related to the current flowing through the transistor, which is directly related to voltage $V_{in}$ applied at the gate. Since the output voltage is taken between the resistor and ground, the output voltage $V_{out}$ is directly related to the input voltage $V_{in}$, which explains why they follow the same trend. It transitions from cutoff to saturation to triode, but does not reach triode in the range of voltages considered. The proper bias point occurs in the middle of the saturation region, for $V_{in} \approx 4.2$\si{\volt}, since the slope is steepest. Moreover, biasing the amplifier in the saturation region leads to high linearity in the output of the amplifier.
