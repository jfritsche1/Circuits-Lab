The common-source amplifier which utilizes the MOSFET has a significantly higher cutoff frequency than the common-emitter amplifier which utilizes the BJT. The common-emitter amplifier would have its cutoff frequency around the order of magnitude of $~100$\si{\kilo\hertz} while the common-source amplifier is found to have its cutoff frequency in the \si{\mega\hertz} range. However, noticeable second-order distortion is observed in input and output signals that approach or exceed the cutoff frequency. This distortion however, is not due to delay times as it is observed that inversion and amplification still occurs at an adequately fast pace around the cutoff frequency. For the common-emitter amplifier, second-order effects are not seen near the cutoff frequency, but switching time is too slow for voltages to fully develop. As a result, the advantages and disadvantages of the MOSFET and BJT are made more clear.